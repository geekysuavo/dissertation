
\chapter{Phase-Scatter Correction of NMR Datasets}

\section{Introduction}

\begin{doublespace}
Normalization applied directly to hypercomplex NMR data (or its real component)
is sub-optimal, as even small phase differences between observations can
frustrate the estimation of normalization factors
(See \hyperlink{section.3.3}{Section 3.3}). Possibly worse, blind
normalization of poorly phased spectral data can accentuate experimentally
irrelevant spectral features in a data tensor during multivariate modeling,
leading the analyst to erroneous conclusions. These difficulties motivated
the development of phase-scatter correction (PSC, \cite{worley:abio2013}) as
a means of simultaneously correcting the coupled phase errors and dilution
errors that are present in hypercomplex NMR data tensors.
\end{doublespace}

\subsection{Metabolomics}

\begin{doublespace}
FIXME.
\end{doublespace}

\subsection{High-throughput Screening}

\begin{doublespace}
FIXME.
\end{doublespace}

\section{Theory}

\subsection{Multiplicative Scatter Correction}

\begin{doublespace}
Phase-scatter correction (PSC) is effectively an extension of multiplicative
scatter correction (MSC) to handle phase errors during normalization. In MSC,
each real spectrum is scaled around its mean intensity and shifted to match a
reference spectrum, typically the mean of the dataset \cite{fearn:cils2009}.
Optimal normalization factors ($\mathbf{b}$) of a data matrix $\mathbf{X}$ are
determined by a linear regression of the mean-centered reference vector onto
the mean-centered matrix:

\begin{equation}
(\mathbf{X} - \bar{\mathbf{X}})^T \mathbf{b} = (\mathbf{r} - \bar{r})^T
\end{equation}

where observations are stored as row vectors in $\mathbf{X}$, and $\mathbf{r}$
is the reference observation row vector. The equation above represents an
overdetermined system of linear equations, therefore the least-squares estimate
of $\mathbf{b}$ may be computed rapidly, and MSC is rather computationally
efficient.
\end{doublespace}

\subsection{Phase-scatter Correction}

\begin{doublespace}
PSC additionally corrects zero- and first-order phase errors during
normalization, requiring a nonlinear optimization of the following objective:

\begin{equation}
Q(\mathbf{X} \mid \mathbf{p})
 = \sum_{n=1}^N \| \mathbf{z}_n \circ \mathbf{x}_n - \mathbf{r} \|_2^2
\end{equation}

where $\circ$ denotes the element-wise product, the mean-centered matrix
$\mathbf{X}$ lies in $\mathbb{H}_1^{N \times K}$, the mean-centered reference
$\mathbf{r}$ lies in $\mathbb{H}_1^K$, and the set of parameters $\mathbf{p}$
includes a normalization factor and two phase errors per observation
in $\mathbf{X}$:

\begin{equation}
\mathbf{p} = \{
  b_1, \dots b_N,
  \theta_{0,1}, \dots \theta_{0,N},
  \theta_{1,1}, \dots \theta_{1,N} \}
\end{equation}

and each vector $\mathbf{z}_n$ contains the normalization and phase corrections
for the $n$-th observation $\mathbf{x}_n$:

\begin{equation}
z_{n,k} = b_n e^{u_1 (\theta_{0,n} + \theta_{1,n} k)}
\end{equation}

Because the reference observation $\mathbf{r}$ is fixed during optimization,
minimization of $Q(\mathbf{X} \mid \mathbf{p})$ may be achieved by
independently minimizing each observation's contributions. Minimization is
carried out for every observation in the data matrix using Levenberg-Marquardt
nonlinear least squares \cite{marquardt:jsiam1963} as implemented by the
{\it leasqr} function in GNU Octave, a function similar to MATLAB's
{\it nlinfit}. Each corrected spectrum $\hat{\mathbf{x}}_n$ is then returned
from optimization as follows:

\begin{equation}
\hat{\mathbf{x}}_n = \mathbf{z}_n \circ \mathbf{x}_n + \bar{r}
\end{equation}

Phase-scatter correction of 50 1D \hnmr{} NMR spectra having 32$k$ complex
points each requires approximately 30 seconds on a single-core 3.2 GHz Intel
workstation running GNU Octave 3.6.
\end{doublespace}

\section{Materials and Methods}

\begin{doublespace}
FIXME.
\end{doublespace}

\section{Results}

\begin{doublespace}
FIXME.
\end{doublespace}

\section{Discussion}

\begin{doublespace}
FIXME.
\end{doublespace}

\section{Conclusions}

\begin{doublespace}
FIXME.
\end{doublespace}

\bibliographystyle{abbrv}
\bibliography{bworley}

