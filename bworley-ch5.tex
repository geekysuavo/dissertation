
\chapter{Phase-Scatter Correction of NMR Datasets}

\section{Introduction}

\begin{doublespace}
Normalization applied directly to hypercomplex NMR data (or its real component)
is sub-optimal, as even small phase differences between observations can
frustrate the estimation of normalization factors
(See \hyperlink{section.3.3.3}{Section 3.3.3}). Possibly worse, blind
normalization of poorly phased spectral data can accentuate experimentally
irrelevant spectral features in a data tensor during multivariate modeling,
leading the analyst to erroneous conclusions. These difficulties motivated
the development of phase-scatter correction (PSC, \cite{worley:abio2013}) as
a means of simultaneously correcting the coupled phase errors and dilution
errors that are present in hypercomplex NMR data tensors.
\end{doublespace}

\subsection{Metabolomics}

\begin{doublespace}
FIXME.
\end{doublespace}

\subsection{High-throughput Screening}

\begin{doublespace}
FIXME.
\end{doublespace}

\section{Theory}

\begin{doublespace}
FIXME.
\end{doublespace}

\section{Materials and Methods}

\begin{doublespace}
FIXME.
\end{doublespace}

\section{Results}

\begin{doublespace}
FIXME.
\end{doublespace}

\section{Discussion}

\begin{doublespace}
FIXME.
\end{doublespace}

\section{Conclusions}

\begin{doublespace}
FIXME.
\end{doublespace}

\bibliographystyle{abbrv}
\bibliography{bworley}

