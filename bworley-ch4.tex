
\chapter{The MVAPACK Suite for NMR Chemometrics}

\section{Introduction}

\begin{doublespace}
The biochemical laboratory procedures involved in metabolomics experiments
are potentially straightforward and inexpensive, depending on the biological
systems and pathways under study \cite{zhang:jiomic2013}. The minimal sample
handling requirements of 1D \hnmr NMR spectroscopy and the immense sensitivity
of multivariate bilinear factorizations such as principal component analysis
(PCA) and partial least squares (PLS) make NMR metabolic fingerprinting
especially attainable. This low barrier to entry has no doubt contributed to
the rapid recent growth of the field. Unfortunately, the data handling tasks
of NMR metabolomics are far more difficult to properly execute. Commercial
software packages available for multivariate analysis (e.g. SIMCA, PLS Toolbox,
The Unscrambler, {\it etc.}) tend to be expensive and require more software for
upstream processing and treatment of spectral data. Analysts are thus required
to first open and process NMR data in packages such as ACD/1D NMR Manager
(Advanced Chemistry Development), Mnova NMR (Mestrelabs Research) and perform
further statistical treatment in MATLAB (The Mathworks, Natick, MA), R, or
Microsoft Excel. This results in an unnecessarily cumbersome and time-consuming
data handling pipeline by forcing the analyst to pass data between multiple
software packages. As a result, the field of metabolomics research is littered
with unpublished ``in-house'' software solutions created for processing,
treating or modeling NMR datasets
\cite{viant:bbrc2003,
      verhoeckx:immpc2004,
      cloarec:anchem2005a,
      cloarec:anchem2005b,
      dieterle:anchem2006,
      kang:food2008,
      wiklund:anchem2008}. This continued reinvention of the wheel impedes
progress in the field and complicates the tasks of standardization and
communication of protocols that the metabolomics community is desperately
attempting to achieve \cite{lindon:nbiot2005,goodacre:metab2007}. Insult is
then added to injury, as these in-house solutions are far less likely than
their commercial counterparts to include proper means of validating trained
multivariate models, further contributing to the general lack of model
validation already present in the field \cite{westerhuis:metab2008}. While
the community has released several official software packages targeted at
metabolomics
\cite{jarvis:binf2006,
      daszykowski:cils2007,
      wang:bmcb2009,
      izquierdo:bmcb2009,
      xia:nar2012,
      gaude:cmb2013,
      alonso:anchem2014}, none provide a complete, well-validated data path.
At the time of this writing, no single software package existed to bring raw
NMR data along its complete journey to validated, interpretable multivariate
models.
\\\\
These issues motivated the development of a free and open-source software
package, MVAPACK, that provides a complete pipeline of functions for NMR
chemometrics and metabolomics. MVAPACK is written in the GNU Octave
mathematical programming language \cite{eaton2008}, which is also open-source
and nearly syntactically identical to MATLAB. Thus, the installation of
GNU/Linux, Octave and MVAPACK onto a commodity workstation provides a uniform
environment in which a data analyst may truly work ``from FIDs to models'' in
a few minutes using a set of well-documented, open-source, high-level data
handling functions.
\end{doublespace}

\section{Materials and Methods}

\subsection{Software Implementation}

\begin{doublespace}
FIXME
\end{doublespace}

\subsection{Feature Set}

\begin{doublespace}
FIXME
\end{doublespace}

\section{Discussion and Conclusions}

\begin{doublespace}
This chapter presents MVAPACK, a completely free and open-source data handling
environment tailor-suited to NMR chemometrics and \hnmr NMR and MS metabolic
fingerprinting applications. Unlike data handling chains composed of multiple
commercial software packages, MVAPACK is free to use, modify and distribute
according to the GNU General Public License \cite{gpl3} and provides a single
consistent data handling environment. Because MVAPACK is written for GNU
Octave, researchers already familiar with MATLAB syntax will also be familiar
with MVAPACK without a considerable learning curve. Datasets and results
obtained using MVAPACK are readily saved and exhanged using GNU Octave built-in
support for the MATLAB {\it mat}-file format.
\\\\
A recent review \cite{izquierdo:pnmrs2011} of software packages targeted at
metabolomics highlights the piecemeal nature of 1D \hnmr NMR data handling in
the field, where no single package is capable of performing all the tasks
required by the analyst. MVAPACK addresses this need by providing a complete
pipeline that is tuned for metabolic fingerprinting. Use of MVAPACK reduces
data analysis time in metabolic fingerprinting from days to minutes, simply
by collecting all the required functions into a single scriptable environment.
In fact, the example script in Figure 4.1 would execute in under five minutes
on a modern GNU/Linux or Mac OS X computer system.
\\\\
The routine processing of {\it any} 1D and 2D NMR spectral data may be readily
done with MVAPACK, and processing routines are easily batched. The MVAPACK
scripts written to analyze the datasets in \hyperlink{chapter.3}{Chapter 3}
are composed of intuitive, modular commands that logically subdivide the script
into recognizable tasks like automatic phase correction, spectral alignment,
normalization, {\it etc}. Furthermore, aside from physical memory limitations
of the host computer, MVAPACK does not impose any limit on the number of NMR
observations that may be simultaneously handled.
\\\\
A major advantage of MVAPACK is the seamless transfer of the processed, treated
NMR data to multivariate statistical analyses. The PCA, PLS, OPLS and LDA
bilinear modeling algorithms, now ubiquitous in the metabolomics community,
are all implemented in MVAPACK using a consistent under-the-hood framework.
Model results may be visualized and interpreted using MVAPACK routines that
provide scatter, line and bar plots of model scores, loadings and validation
statistics. Critically, MVAPACK automatically ensures that {\it all} trained
models are valid using leave-one-out and Monte Carlo leave-$n$-out internal
cross-validation routines and provides further means of validating supervised
models in the form of CV-ANOVA and response permutation significance testing.
Several powerful examples of MVAPACK applied to real datasets are presented in
\hyperlink{chapter.3}{Chapter 3}. Because it implements well-established
algorithms available from peer-reviewed chemometrics literature, MVAPACK
generates identical results when compared to expensive software packages
like Umetrics SIMCA-P+.
\\\\
In short, MVAPACK provides a complete platform for NMR chemometric data
handling that is ideal for both routine handling of metabolomics datasets and
development of novel algorithms. Unlike its closed-source predecessors, the
modular, open-source design of MVAPACK readily accepts new functionality,
allowing it to grow and maintain pace with the state-of-the-art in the
chemometrics literature. MVAPACK is freely available for download at
\url{http://bionmr.unl.edu/mvapack.php}. Detailed documentation of MVAPACK,
all datasets presented in \hyperlink{chapter.3}{Chapter 3}, and the scripts
used to handle them are also available for download.
\end{doublespace}

\bibliographystyle{abbrv}
\bibliography{bworley}

