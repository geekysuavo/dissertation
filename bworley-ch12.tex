
\chapter{Summary and Future Directions}

\begin{quote}
{\it
  Chemists, in particular, cannot understand why they should fund
  someone to do data analysis.}
\\\\
 -- Richard Brereton \cite{brereton:jchemo2014b}
\end{quote}

\section{The Need for Chemometrics}

\begin{doublespace}
The field of chemometrics is still in its infancy, but the chemometric
practice of extracting quantitative chemical information from data collected
on complex samples is much older, and has innumerable applications in
chemistry \cite{wold:cils1995,brereton:jchemo2014b}. While the standard
toolbox of $t$-tests, run charts and univariate distributions has served
analytical chemists well, the analysis of spectral measurements of
multi-component mixtures demands a more computationally intensive
approach.
\\\\
However, optimal chemometric modeling of spectral data does not begin when
the data are read in for the first time, but before acquisition has even
been performed. Successful experimental design relies on data collection
procedures that yield informative, high-quality measurement results. Spectra
having the highest possible resolution, dynamic range and signal-to-noise
ratio are necessary if reliable conclusions are to be drawn from their models.
In multidimensional NMR experiments, methods of sparse data collection are
becoming increasingly popular, as they provide avenues for maximizing spectral
quality. In these nonuniform sampling (NUS) methods, the greatest contributing
factor to spectral quality is the sampling scheme, and the generation of
sampling schemes that optimize various spectral features (i.e. sensitivity
or resolution) is still an active area of fundamental research
\cite{mobli:jmr2015}. \hyperlink{chapter.2}{Chapter 2} introduces a general
framework for multidimensional nonuniform sampling that extends the work of
Hyberts and Wagner \cite{hyberts:jacs2010} and deterministically generates
nonuniform sampling schedules that perform as well or better than stochastic
methods \cite{worley:jmr2015}. By suggesting an alternative mechanism for
introducing irregularity into a sampling schedule, burst-augmented gap
sampling aims to provoke further investigation into which features of a
sampling schedule yield optimal spectral results. Furthermore, this new
framework is the first proposed mechanism for deterministically constructing
sampling schedules on a multidimensional Nyquist grid \cite{eddy:jmr2012}
based on a general equation.
\\\\
The processing, treatment and modeling of spectral measurements using
multivariate statistics, outlined in \hyperlink{chapter.3}{Chapter 3},
is a nuanced task, with many pitfalls awaiting the chemist who lacks
experience and training in multivariate data analysis. Most applications
of chemometrics are performed by analytical chemists, whose expertise lies
with a certain type of instrumentation rather than statistics. In order to
promote proper, statistically sound data handling practices, chemometricians
must begin to place easy-to-use, well-documented software packages in the
hands of chemists. These software packages must simultaneously provide
powerful mechanisms of multivariate data analysis, educate users about
proper data handling, and encourage further extension and collaboration
between fundamentally focused chemometricians and applications-driven
chemists. \hyperlink{chapter.5}{Chapter 5} introduces MVAPACK
\cite{worley:acscb2014}, an open-source suite of simple GNU Octave
\cite{eaton2008} functions that aims to address those goals, and
\hyperlink{chapter.4}{Chapter 4} describes its use on a wide variety of
applications within the rapidly growing field of metabolomics.
\end{doublespace}

\bibliographystyle{abbrv}
\bibliography{bworley}

