
\begin{abstract}
\begin{doublespace}
The amount of information collected and analyzed in biochemical and
bioanalytical research has exploded over the last few decades, due in large
part to the increasing availability of analytical intrumentation that yields
information-rich spectra. Datasets from Nuclear Magnetic Resonance (NMR),
Mass Spectrometry (MS), infrared (IR) or Raman spectroscopy may easily carry
tens to hundreds of thousands of potentially correlated variables observed
from only a few samples, making the application of classical statistical
methods inappropriate, if not impossible. Drawing useful biochemical
conclusions from these unique sources of data requires the use of specialized
multivariate data handling techniques.
\\\\
Unfortunately, proper implementation of many new multivariate algorithms
requires domain knowledge in mathematics, statistics, digital signal
processing, and software engineering {\it in addition to} analytical chemical
and biochemical expertise. As a consequence, analysts using multivariate
statistical methods were routinely required to chain together multiple
commercial software packages and fashion small ad hoc software solutions
to interpret a single dataset. This has been especially true in the field of
NMR metabolomics, where no single software package, free or otherwise, was
capable of completing all operations required to transform raw instrumental
data into a set of validated, informative multivariate models. Therefore,
while many powerful methods exist in published literature to statistically
treat and model multivariate spectral data, few are readily available for
immediate use by the community as a whole.
\\\\
This dissertation describes the development of an end-to-end software solution
for the handling and multivariate statistical modeling of spectroscopic data,
called MVAPACK, and a set of novel spectral data acquisition, processing and
treatment algorithms whose creation was expedited by MVAPACK. A final foray
into the potential existence of \npistar{} interactions within proteins is
also presented.
\end{doublespace}
\end{abstract}

